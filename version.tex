\section{Data Versioning}

Versioning is necessary if the application wants to update previously published data and publish again under the same name. Since NDN mandates every data packet to have a unique name, a version component must be added to allow unambiguous naming for different versions of the data. The version component encodes the version number using application-specific format. The current convention used by NDN Repo is to start the version component with a special \textbf{FD marker}, followed by a big-endian binary NDN timestamp (see NDN packet format specification), and place it as the next-to-last component in the name (not counting the implicit digest). Applications should follow this convention if they want interaction with repos.

Applications with special requirement on data versioning may also define their own versioning semantics. Typically, version numbers used by NDN applications fall into two categories: timestamp or counter. Timestamp-based versions are generated from the system clock to indicate the creation time of the data (applications may choose any time unit suitable for their functionalities), while counter-based versions usually reflect the number of previous versions. The encoding of the version component may or may not use the special markers. However, it is strongly recommended that the encoded version component is monotonically increasing in the NDN canonical ordering. That is, the newer versions should always come after the older versions.


%%% Local Variables: 
%%% mode: latex
%%% TeX-master: "convention"
%%% End: 
